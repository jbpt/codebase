\section*{Stochastic Conformance Checking Measures}
\setcounter{subsection}{0}
\subsection{Stochastic Precision and Recall Measures}
To compute the stochastic precision value between the event log and the process model, use the option (\textcolor{darkcandyapplered}{\footnotesize\ttfamily-sp}) as follows.
\begin{lstlisting}[style=CL]
>java -jar !jbpt-pm-entropia-1.5.jar! @-sp@ & -rel=log.xes& &-ret=automaton.json&
\end{lstlisting}
\textbf{Output Screen:}%chnage
\lstinputlisting[style=DOS]{screens/screen_(-sp).txt}

Use the option (\textcolor{darkcandyapplered}{\footnotesize\ttfamily-sr}) instead of (\textcolor{darkcandyapplered}{\footnotesize\ttfamily-sp}) in order to get the stochastic recall value between the event log and process model. 
\begin{lstlisting}[style=CL]
>java -jar !jbpt-pm-entropia-1.5.jar! @-sr@ & -rel=log.xes& &-ret=automaton.json&
\end{lstlisting}
\textbf{Output Screen:}%chnage
\lstinputlisting[style=DOS]{screens/screen_(-sr).txt}
\subsection{Entropic Relevance Measure}
You can measure relevance of a stochastic process model to an event log using the option (\textcolor{darkcandyapplered}{\footnotesize\ttfamily-r}), as the following command shows. Note that the retrieved model is specified to the stochastic process model, i.e. the stochastic deterministic finite automaton (SDFA), in JSON format.

\begin{lstlisting}[style=CL]
>java -jar !jbpt-pm-entropia-1.5.jar! @-r@ &-rel=&log.xes &-ret=&automaton.json
\end{lstlisting}
\textbf{Output Screen:}%chnage
\lstinputlisting[style=DOS]{screens/screen_(-r).txt}
