\section*{Models and Event Log}
The tutorial will use example files that are provided with the tool, which are the event log coded in XES format (log.xes), process model modelled as a Petri Net (model.pnml) and stochastic process model as an SDFA coded in JSON format (automaton.json). Table 1, Figure 1 and Figure 2 represent the three files, respectively. 


\begin{figure}[h!]
\hspace{3mm}
 \begin{minipage}{0.45\textwidth}
\begin{center}
\vspace{-3mm}
\begin{tikzpicture}[scale=0.6, transform shape, ->, >=stealth', shorten >=1pt, auto, node distance=20mm, on grid, semithick, transition/.style={fill=rebeccapurple!30, draw, minimum size=9mm, drop shadow}, place/.style={fill=black!5, draw, circle, minimum size=9mm, drop shadow}]

\node[place,label=90:\small $p_0$]				(p0) 											{\Large $\bullet$};
\node[transition,label=90:\small $t_0$]		(t0) [right=of p0] 				{\Large $\texttt{a}$};
\node[place,label=90:\small $p1$] 				(p1) [above right=of t0]	{};
\node[place,label=270:\small $p_2$] 			(p2) [below right=of t0]	{};
\node[transition,label=90:\small $t_1$] 	(t1) [right=of p1]				{\Large $\texttt{b}$};
\node[transition,label=270:\small $t_3$] 	(t3) [right=of p2]				{\Large $\texttt{c}$};
\node[transition,label=0:\small $t_2$] 		(t2) [below right=of p1,xshift=17pt]				{\Large $\texttt{d}$};
\node[place,label=90:\small $p3$] 				(p3) [right=of t1]				{};
\node[place,label=270:\small $p4$] 				(p4) [right=of t3]				{};
\node[transition,label=90:\small $t_4$] 	(t4) [below right=of p3]	{\Large $\texttt{e}$};
\node[place,label=90:\small $p5$] 				(p5) [right=of t4]				{};

\path (p0) edge node {} (t0)
			(t0) edge node {} (p1)
					 edge node {} (p2)
			(p1) edge node {} (t1)
			(p2) edge node {} (t3)
			(t1) edge node {} (p3)
			(t3) edge node {} (p4)
			(t2) edge node {} (p1)
					 edge node {} (p2)
			(p3) edge node {} (t2)
					 edge node {} (t4)
			(p4) edge node {} (t2)
					 edge node {} (t4)
			(t4) edge node {} (p5)
;
\end{tikzpicture}
\vspace{-2mm}
\caption{Process Model.}
\label{fig:petri:net}
\vspace{-2mm}
\end{center}
\end{minipage}
\hspace{7mm}
\begin{minipage}{0.45\textwidth}
\begin{center}
\vspace{-9mm}
\begin{tikzpicture}[scale=0.65, transform shape, ->, >=stealth', shorten >=1pt, auto, initial text=, node distance=20mm, on grid, semithick, every state/.style={fill=rebeccapurple!30, draw, circular drop shadow, text=black, minimum size=9mm}]
\node[initial,state,label=270:$s_0$]	(s0) 											{$0$};
\node[state,label=270:$s_1$]					(s1) [right=of s0] 				{$0$};
\node[state,label=0:$s_2$]						(s2) [above right=of s1]	{$0$};
\node[state,label=270:$s_3$] 					(s3) [below right=of s2]	{$0$};
\node[state,label=0:$s_4$] 					  (s4) [below right=of s1]	{$\nicefrac{1}{5}$};
\node[state,label=270:$s_5$] 				  (s5) [right=of s3]				{$1$};

\path (s0) edge node {$\texttt{a}(1)$} (s1)
			(s1) edge node {$\texttt{b}(\nicefrac{1}{2})$} (s2)
			     edge node[below, xshift=-14pt, yshift=-2pt] {$\texttt{c}(\nicefrac{1}{2})$} (s4)
			(s2) edge node {$\texttt{c}(1)$} (s3)
			(s3) edge node[above]  {$\texttt{d}(\nicefrac{1}{2})$} (s1)
					 edge node {$\texttt{e}(\nicefrac{1}{2})$} (s5)
			(s4) edge node[below, xshift=16pt, yshift=-2pt] {$\texttt{b}(\nicefrac{4}{5})$} (s3);
\end{tikzpicture}
\vspace{3.5mm}
\caption{Stochastic Process Model.}
\label{fig:SDFA}
\vspace{-10mm}
\end{center}
\end{minipage}

\begin{center}
\vspace{5mm}
\begin{minipage}{0.45\textwidth}
 \centering
\includegraphics[width=0.40\textwidth]{fig/eventLog.pdf} 
\caption*{ Table 1: Event Log}
\label{fig:Log}
\end{minipage}
\end{center}
\vspace{-4mm}
\end{figure}

All files are located in the folder \textbf{jbpt-pm\slash entropia\slash examples\slash}. If the folder does not contain the files, you can download them from \url{https://github.com/jbpt/codebase/tree/master/jbpt-pm/entropia/examples}. It is recommended to use the event log and models provided to understand how to run the \textit{Entropia} commands. Then, try it with your own data. Refer to Table III in the demo paper for more details about file types and formats supported by the tool.
