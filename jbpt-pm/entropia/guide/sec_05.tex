\section*{Bootstrap Generalization Measure}
\setcounter{subsection}{0}

To compute the generalization of a process model with bootstrap generalization approach~\cite{Polyvyanyy2022Bootstrapping}, use the option (\textcolor{darkcandyapplered}{\footnotesize\ttfamily-bgen}) as follows.
\begin{lstlisting}[style=CL]
>java -jar !jbpt-pm-entropia-1.5.jar! @-bgen@ &-rel&=log3.xes &-ret&=model3.json
\end{lstlisting}

\textbf{Output Screen:}%chnage
\lstinputlisting[style=DOS]{screens/screen_(-bgen1).txt}

Note that the model you want to analyze should be provided in JSON format.
Each DFG node has a string label for the activity name, a numerical frequency for activity execution, and a unique identifier.
Arcs connect nodes by their source and target numbers, along with their occurrence frequency.
The start and end nodes should be labeled \texttt{INPUT} and \texttt{OUTPUT} to indicate process boundaries.
Refer to Listing~\ref{lst:dfg} for a sample representation.
%\begin{wrapfigure}{R}{0.35\textwidth}
%  \vspace{-5mm}

\begin{minipage}[t]{\textwidth}
        \vspace{0pt}
        \begin{lstlisting}[caption={Sample JSON representation.}, label={lst:dfg}]
{ "nodes": [
    { "id": 1, "label": "INPUT",  "freq": 30 },
    { "id": 2, "label": "A",      "freq": 30 },
    { "id": 3, "label": "B",      "freq": 30 },
    { "id": 4, "label": "OUTPUT", "freq": 30 }
  ],
  "arcs": [
    { "from": 1, "to": 2, "freq": 30 },
    { "from": 2, "to": 3, "freq": 30 },
    { "from": 3, "to": 4, "freq": 18 }
  ]
}
        \end{lstlisting}
    \end{minipage}
%    \vspace{-5mm}
%\end{wrapfigure}

You can adjust the underlying bootstrapping process with several parameters for bootstrap generalization estimation, including sample size (\texttt{n}), number of samples (\texttt{m}), number of log generations (\texttt{g}), crossover subtrace length (\texttt{k}), breeding probability (\texttt{p}), and threshold for confidence interval of bootstrap samples (\texttt{ep}).
These parameters, detailed in~\cite{Polyvyanyy2022Bootstrapping}, are optional for tool usage.

\begin{lstlisting}[style=CL]
>java -jar !jbpt-pm-entropia-1.5.jar! @-bgen@ &-rel&=log3.xes &-ret&=model3.json -s &-m&=1000
&-ep&=0.005
\end{lstlisting}

\textbf{Output Screen:}%chnage
\lstinputlisting[style=DOS]{screens/screen_(-bgen2).txt}

Here, bootstrapping continues until 1,000 samples
are reached, unless the confidence interval for both
precision and recall of the bootstrap samples falls
below 0.005, triggering early termination.
And this calculation will  be done in silent mode,
showing only the final result.

%Model-system precision: 0.922949821721433 +/- 0.00469559925550323
%Model-system recall: 0.923083213341255 +/- 0.004965976603082337

Another configuration might be as follows:

\begin{lstlisting}[style=CL]
>java -jar !jbpt-pm-entropia-1.5.jar! @-bgen@ &-rel&=log3.xes &-ret&=model3.json &-n&=1000 &-m&=20
&-p&=0.5
\end{lstlisting}

\textbf{Output Screen:}%chnage
\lstinputlisting[style=DOS]{screens/screen_(-bgen3).txt}

This configuration generates 20 bootstrap samples, each
containing 1,000 traces, with new traces being generated
50\% of the time, and existing traces used the remaining 50\%.








